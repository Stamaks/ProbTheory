\documentclass[]{article}

\usepackage[utf8x]{inputenc}
\usepackage{graphicx}
\graphicspath{ {./images/} }
\usepackage[russian]{babel}
\usepackage{amsmath}
\usepackage{amssymb}
\usepackage[a4paper, total={6.5in, 9in}]{geometry}
\usepackage{array}
\newcolumntype{C}[1]{>{\centering\let\newline\\\arraybackslash\hspace{0pt}}m{#1}}
\newcommand{\tabitem}{~~\llap{\textbullet}~~}

\begin{document}
	\begin{figure}[t]
		\centering
		\fontseries{b}
		\large
		НАЦИОНАЛЬНЫЙ ИССЛЕДОВАТЕЛЬСКИЙ УНИВЕРСИТЕТ\\
		«ВЫСШАЯ ШКОЛА ЭКОНОМИКИ»\\
		Дисциплина: «Теория вероятностей и математическая статистика»
	\end{figure}
	
	\begin{figure}[h]
	\vspace{3in}
	\fontseries{b}
	\centering
	\Huge
	\textbf{Домашнее задание 4}\\
	Вариант 15 
	\end{figure}
	
	\vspace{2in}
	\Large
	\raggedleft
	Выполнила: Карнаухова Алена,\\
	студентка группы 172\\
	\vspace{12pt}
	Преподаватель: Горяинова Е.Р.,\\
	доцент департамента матматики\\
	факультета экономических наук
	
	\begin{figure}[b]
		\centering
		Москва \the\year
	\end{figure}
	
	\thispagestyle{empty}
	
	\newpage
	
	\centering
	\Large
	\textbf{Задача 1.}
	
	\vspace{10pt}
	
	\raggedright
	
	По результатам 30-ти измрения скорости $v$ получена оценка дисперсии $\tilde{\sigma}^2_v = 6 $ м$^2 / c^2$. Построить $90$ процентный доверительный интервал для неизвестных величин  - дисперсии $\sigma^2_v$ и среднего квадратичного отклонения $\sigma_v$, считая величину $v$ распределенной по нормальному закону.
	
	\vspace{20pt}
	
	По теореме Фишера случайная величина $\xi$
	
	\vspace{10pt}
	
	$$\xi = \frac{(n-1) \tilde{\sigma}^2_v}{\sigma^2}$$\\
	
	\vspace{10pt}
	
	имеет распределение $\chi^2(n-1)$. Тогда:\\
	
 $$P \left(  \chi^2_{\frac{1-\alpha}{2},n-1}  \leqslant \xi \leqslant\chi^2_{\frac{1+\alpha}{2},n-1}\right) =\alpha$$
 
 	Подставим $\xi$ и получим:\\
 	
	 $$P \left(  \chi^2_{\frac{1-\alpha}{2},n-1}  \leqslant \frac{(n-1) \tilde{\sigma}^2_v}{\sigma^2} \leqslant\chi^2_{\frac{1+\alpha}{2},n-1}\right) =\alpha$$
	 
	 $$P \left( \frac{1}{\chi^2_{\frac{1+\alpha}{2},n-1}}  \leqslant \frac{\sigma^2}{(n-1)\tilde{\sigma}^2_v} \leqslant \frac{1}{\chi^2_{\frac{1-\alpha}{2},n-1}}\right) =\alpha$$	
	 	
	 $$P \left( \frac{\tilde{\sigma}^2_v(n-1)}{\chi^2_{\frac{1+\alpha}{2},n-1}}  \leqslant \sigma^2 \leqslant \frac{\tilde{\sigma}^2_v(n-1)}{\chi^2_{\frac{1-\alpha}{2},n-1}}\right) =\alpha$$	
	 
	 \vspace{10pt}
	 
	 Подставим значения в формулу:
	 
	 $$P \left( \frac{6 \cdot 29}{\chi^2_{0.95,29}}  \leqslant \sigma^2 \leqslant \frac{6 \cdot 29}{\chi^2_{0,05,29}}\right) =0,9$$	
	 
	 $$P \left( \frac{174}{42,557}  \leqslant \sigma^2 \leqslant \frac{174}{17,708}\right) =0,9$$	
	 
	 $$P \left( 4,089 \leqslant \sigma^2 \leqslant 9,826\right) =0,9$$	
	 
	 \textbf{Ответ:} $\sigma^2 \in \left[ 4,089 ; 9,826\right]$ 
	
	
\end{document}