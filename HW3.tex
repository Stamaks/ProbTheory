\documentclass[]{article}

\usepackage[utf8x]{inputenc}
\usepackage{graphicx}
\graphicspath{ {./images/} }
\usepackage[russian]{babel}
\usepackage{amsmath}
\usepackage{amssymb}
\usepackage[a4paper, total={6.5in, 9in}]{geometry}
\usepackage{array}
\newcolumntype{C}[1]{>{\centering\let\newline\\\arraybackslash\hspace{0pt}}m{#1}}
\newcommand{\tabitem}{~~\llap{\textbullet}~~}

\begin{document}
	\begin{figure}[t]
		\centering
		\fontseries{b}
		\large
		НАЦИОНАЛЬНЫЙ ИССЛЕДОВАТЕЛЬСКИЙ УНИВЕРСИТЕТ\\
		«ВЫСШАЯ ШКОЛА ЭКОНОМИКИ»\\
		Дисциплина: «Теория вероятностей и математическая статистика»
	\end{figure}
	
	\begin{figure}[h]
	\vspace{3in}
	\fontseries{b}
	\centering
	\Huge
	\textbf{Домашнее задание 3}\\
	Вариант 15 
	\end{figure}
	
	\vspace{2in}
	\Large
	\raggedleft
	Выполнила: Карнаухова Алена,\\
	студентка группы 172\\
	\vspace{12pt}
	Преподаватель: Горяинова Е.Р.,\\
	доцент департамента матматики\\
	факультета экономических наук
	
	\begin{figure}[b]
		\centering
		Москва \the\year
	\end{figure}
	
	\thispagestyle{empty}
	
	\newpage
	
	\centering
	\large
	\textbf{Задача 1.}
	
	\vspace{10pt}
	
	\raggedright

	Случчайная величина $(\xi, \eta)$ распределена по нормальному закону с мат. ожиданием $(\mu_1, \mu_2)$ и ковариационной матрицей 
	$\sum = 
 \begin{pmatrix}
  \sigma_{\xi}^2 & cov(\xi; \eta) \\
  cov(\eta; \xi) & \sigma_{\eta}^2 
 \end{pmatrix}$. Найти: $P\{ \eta > 2\xi\}$ при $(\mu_1, \mu_2) = (2;7)$, $\sum = 
 \begin{pmatrix}
  4 & -1 \\
  -1 & 16 
 \end{pmatrix}$.
 
 \vspace{20pt}
 
 $P\{\eta > 2 \xi\} = P\{2 \xi - \eta < 0\} = ?$
 
\vspace{10pt}

Случайная величина  $(2 \xi - \eta)$ распределена нормально, так как является алгебраической суммой случайных величин, распределенных нормально.

\vspace{10pt}

$E(2\xi - \eta) = 2E\xi - E\eta = 2 \cdot 2 - 7 = -3;$

\vspace{10pt}

$D(2\xi - \eta) = 4D\xi + D\eta + 2cov(2\xi, -\eta) =  4D\xi + D\eta - 4cov(\xi, \eta) = 4 \cdot 4 + 16 - 4 \cdot (-1) = 36;$

\vspace{10pt}

Откуда:

\vspace{10pt}

$(2\xi - \eta) \sim N(-3, 6^2).$

\vspace{10pt}

$P\{2 \xi - \eta < 0\} = \Phi (\frac{0 - (-3)}{6}) = \Phi(\frac{1}{2}) = 0,5 + \Phi_0(\frac{1}{2}) = 0,5 + 0,1915 = 0,6915.$
 

	\vspace{20pt}
	
	\textbf{Ответ:} 0,6915;
	
	\newpage
	
	\centering
	\textbf{Задача 2.}
	
	\vspace{10pt}
	
	\raggedright
	\large
	
	\includegraphics[scale=0.7]{2.png}
	
	\vspace{20pt}
	
	\textbf{1. Вариационный ряд:}\\
	
	\vspace{10pt}
	
	13.13, 13.20, 13.20, 13.23, 13.24, 13.26, 13.28, 13.29, 13.29, 13.29, 13.29, 13.29, 13.29, 13.30, 13.30, 13.31, 13.32, 13.32, 13.32, 13.33, 13.33, 13.33, 13.34, 13.34, 13.34, 13.35, 13.36, 13.37, 13.37, 13.38, 13.38, 13.38, 13.39, 13.39, 13.39, 13.39, 13.40, 13.40, 13.40, 13.40, 13.40, 13.40, 13.40, 13.40, 13.41, 13.42, 13.42, 13.42, 13.43, 13.43, 13.44, 13.44, 13.44, 13.45, 13.45, 13.45, 13.46, 13.46, 13.46, 13.46, 13.47, 13.47, 13.47, 13.48, 13.48, 13.48, 13.48, 13.48, 13.48, 13.48, 13.50, 13.50, 13.50, 13.51, 13.51, 13.51, 13.51, 13.51, 13.52, 13.52, 13.52, 13.53, 13.53, 13.53, 13.53, 13.54, 13.54, 13.55, 13.56, 13.56, 13.56, 13.56, 13.57, 13.57, 13.57, 13.58, 13.58, 13.59, 13.62, 13.62, 13.62, 13.63, 13.64, 13.69.\\
	
	\vspace{10pt}
	
	\textbf{2. Количество интервалов разбиения выборки по формуле Стерджесса:}
	
	\vspace{10pt}
	
	$$n = 1 + 3,322 \cdot lg N$$ где $N $ - это размер выборки.
	
	\vspace{10pt}
	
	$n = 1 + 3,322 \cdot [log_{10} 104] = 7$ \\
	
	\vspace{10pt}
	
	\textbf{3. Таблица сатистического ряда}
	
	\vspace{10pt}
	
	\includegraphics[scale=0.6]{3.png}
	
	\vspace{10pt}
	
	\textbf{4. Гистограмма частот.}
	
	\vspace{10pt}
	
	\includegraphics[scale=0.45]{4.png}
	
	\vspace{10pt}
	
	\textbf{5. Реализации точечных оценок мат. ожидания и деспирсии.}
	
	\vspace{10pt}
	
	Мат. ождидание:
	
	\vspace{10pt}
	
	$E = \frac{1}{n} \sum\limits_{i=1}^n x_i = 13,435.$\\
	
	\vspace{10pt}
	
	Дисперсия:
	
	\vspace{10pt}
	
	$D =  \frac{1}{n} \sum\limits_{i=1}^n (E - x_i)^2 = 0,1097.$
	
	\vspace{10pt}
	
	\textbf{6. По виду гистограммы можно сделать предположение о том, что случайная величина распределена нормально.}
	
\end{document}