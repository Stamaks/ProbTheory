\documentclass[]{article}

\usepackage[utf8x]{inputenc}
\usepackage[russian]{babel}
\usepackage{amsmath}
\usepackage{amssymb}
\usepackage[a4paper, total={6.5in, 9in}]{geometry}
\usepackage{array}
\newcolumntype{C}[1]{>{\centering\let\newline\\\arraybackslash\hspace{0pt}}m{#1}}
\newcommand{\tabitem}{~~\llap{\textbullet}~~}

\begin{document}
	\begin{figure}[t]
		\centering
		\fontseries{b}
		\large
		НАЦИОНАЛЬНЫЙ ИССЛЕДОВАТЕЛЬСКИЙ УНИВЕРСИТЕТ\\
		«ВЫСШАЯ ШКОЛА ЭКОНОМИКИ»\\
		Дисциплина: «Теория вероятностей и математическая статистика»
	\end{figure}
	
	\begin{figure}[h]
	\vspace{3in}
	\fontseries{b}
	\centering
	\Huge
	\textbf{Домашнее задание 1}\\
	Вариант 14 
	\end{figure}
	
	\vspace{2in}
	\Large
	\raggedleft
	Выполнила: Карнаухова Алена,\\
	студентка группы 172\\
	\vspace{12pt}
	Преподаватель: Горяинова Е.Р.,\\
	доцент департамента матматики\\
	факультета экономических наук
	
	\begin{figure}[b]
		\centering
		Москва \the\year
	\end{figure}
	
	\thispagestyle{empty}
	
	\newpage
	
	\centering
	\textbf{Задача 2.}
	
	\vspace{10pt}
	
	\raggedright
	\large
	
	Вероятность выигрыша по лотерейному билету равна P = 0,1. Сколько билетов нужно приобрести, чтобы выигрыш был гарантирован с вероятностью PT = 0,9?
	
	\vspace{20pt}
	
	Вероятность ни разу не выиграть для $n$ попыток составляет $(0,9)^n$. Тогда вероятность обратного события (выиграть хотя бы раз, сделав $n$ попыток) составляет $1 - (0,9)^n$. Выигрыш должен быть гарантирован с вероятностью $0,9$. Тогда из следующего уравнения найдем количество билетов, которое нужно приобрести, чтобы такая вероятность была гарантирована.
	
	\vspace{10pt}
	
	\begin{equation}
	1 - (0,9)^n \geq 0,9;
	\end{equation}
	
	\begin{equation}
	1 - 0,9 \geq (0,9)^n;
	\end{equation}
	
	\begin{equation}
	0,1 \geq (0,9)^n;
	\end{equation}
	
	\vspace{10pt}
	Дальше будем действовать методом подбра и выясним, что подходящее $n = 22.$
	
	\vspace{10pt}
	
	\textbf{Ответ:} 22;
	
	
\end{document}