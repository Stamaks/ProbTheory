\documentclass[]{article}

\usepackage[utf8x]{inputenc}
\usepackage[russian]{babel}
\usepackage{amsmath}
\usepackage{amssymb}
\usepackage[a4paper, total={6.5in, 9in}]{geometry}
\usepackage{array}
\newcolumntype{C}[1]{>{\centering\let\newline\\\arraybackslash\hspace{0pt}}m{#1}}
\newcommand{\tabitem}{~~\llap{\textbullet}~~}

\begin{document}
	\begin{figure}[t]
		\centering
		\fontseries{b}
		\large
		НАЦИОНАЛЬНЫЙ ИССЛЕДОВАТЕЛЬСКИЙ УНИВЕРСИТЕТ\\
		«ВЫСШАЯ ШКОЛА ЭКОНОМИКИ»\\
		Дисциплина: «Теория вероятностей и математическая статистика»
	\end{figure}
	
	\begin{figure}[h]
	\vspace{3in}
	\fontseries{b}
	\centering
	\Huge
	\textbf{Домашнее задание 1}\\
	Вариант 14 
	\end{figure}
	
	\vspace{2in}
	\Large
	\raggedleft
	Выполнила: Карнаухова Алена,\\
	студентка группы 172\\
	\vspace{12pt}
	Преподаватель: Горяинова Е.Р.,\\
	доцент департамента матматики\\
	факультета экономических наук
	
	\begin{figure}[b]
		\centering
		Москва \the\year
	\end{figure}
	
	\thispagestyle{empty}
	
	\newpage
	
	\centering
	\textbf{Задача 1.}
	
	\vspace{10pt}
	
	\raggedright
	\large

	По каналу связи передаются 10 сигналов (вероятность искажения каждого
из них одинакова). Из-за помех 4 из переданных сигналов при приеме искажаются. Какова
вероятность того, что из четырех любых принятых сигналов хотя бы один – искаженный?	

	\vspace{20pt}
	
	Всего мы выбираем 4 любых сигнала из 10. Это мы можем сделать $C_{10}^4$ способами. Нам нужно выбрать случаи, когда хотя бы один из сигналов будет искаженным, но это сложно. Найдем вероятность обратного события (ни один из 4 выбранных сигналов не искажен) и вычтем ее из 1. Неискаженных сигналов всего $10-4$ и пособов выбрать 4 из них всего $C_{10-4}^4 = C_{6}^4$. Получим:
	
	\begin{equation}
	A = 1 - B = 1 - \frac{C_6^4}{C_{10}^4} = 1 - \frac{6!4!6!}{2!4!10!} = 1 - \frac{1}{14} = \frac{13}{14} \approx 0,929;
	\end{equation}
	
	Где $A$ - вероятность искомого события, $B$ - вероятность обратного события.
	
	\vspace{10pt}
	
	\textbf{Ответ:} 0,929;
	
	\vspace{30pt}
	
	\centering
	\textbf{Задача 2.}
	
	\vspace{10pt}
	
	\raggedright
	\large
	
	Вероятность выигрыша по лотерейному билету равна P = 0,1. Сколько билетов нужно приобрести, чтобы выигрыш был гарантирован с вероятностью PT = 0,9?
	
	\vspace{20pt}
	
	Вероятность ни разу не выиграть для $n$ попыток составляет $(0,9)^n$. Тогда вероятность обратного события (выиграть хотя бы раз, сделав $n$ попыток) составляет $1 - (0,9)^n$. Выигрыш должен быть гарантирован с вероятностью $0,9$. Тогда из следующего уравнения найдем количество билетов, которое нужно приобрести, чтобы такая вероятность была гарантирована.
	
	\vspace{10pt}
	
	\begin{equation}
	1 - (0,9)^n \geq 0,9;
	\end{equation}
	
	\begin{equation}
	1 - 0,9 \geq (0,9)^n;
	\end{equation}
	
	\begin{equation}
	0,1 \geq (0,9)^n;
	\end{equation}
	
	\vspace{10pt}
	Дальше будем действовать методом подбра и выясним, что подходящее $n = 22.$
	
	\vspace{10pt}
	
	\textbf{Ответ:} 22;
	
	
\end{document}